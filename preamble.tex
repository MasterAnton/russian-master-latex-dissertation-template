\documentclass[a4paper, 14pt]{extreport}

%\usepackage{showframe} % Отображение границ отступов

\sloppy

% Таблицы
\usepackage{array}
\usepackage{tabularx}
\usepackage{dcolumn}
\usepackage{longtable}
\usepackage{adjustbox}
\usepackage{multirow}

% Кодировка и язык
\usepackage[utf8]{inputenc}
\usepackage[T1,T2A]{fontenc} 
\usepackage[english,russian]{babel}

\usepackage{amsmath}

% Геометрия страницы и графика
\usepackage[left=3cm, right=15mm, top=2cm, bottom=2cm]{geometry} % поля страницы
\usepackage{graphicx}
\usepackage{pdfpages}
\graphicspath{{images/}}

% Прочее
\usepackage[font={normal}]{caption} % настройка подписей к рисункам и таблицам
\usepackage[onehalfspacing]{setspace} % полуторный интервал
\setlength{\parskip}{0cm}
\setlength{\parindent}{1.25cm} 
\usepackage{indentfirst}
\usepackage{hyperref}
\usepackage{float}

% Подписи
\usepackage{caption} %заголовки плавающих объектов
\captionsetup[figure]{name=Рисунок}
\usepackage{ragged2e}
\usepackage{ccaption}
\captiondelim{ — }
\captionstyle{\centering}

% Список источников
\usepackage{csquotes}        
\usepackage[
backend=biber,
hyperref=auto,
sorting=none, % сортировка в порядке встречаемости ссылок
language=auto,
citestyle=gost-numeric,
bibstyle=gost-numeric
]{biblatex}
\addbibresource{biblio.bib}

%Счетчики
\newcounter{mycitecount}  
\usepackage[figure,table,mycitecount]{totalcount}
\usepackage{lastpage}
\AtEveryBibitem{\stepcounter{mycitecount}}

% Замена шрифтов
\usepackage{tempora}

% Настройка подписей к изображениям и таблицам
\captionsetup{format=hang,labelsep=period}

% Настройка заголовков
\usepackage{titlesec}
\titleformat{\chapter}[block]
	{\indent\indent\normalfont\bfseries}{\thechapter}{0.3em}{}
\titleformat{\section}[block]
	{\indent\indent\normalfont\bfseries}{\thesection}{0.3em}{}
\titleformat{\subsection}[block]
	{\indent\indent\normalfont}{\thesubsection}{0.3em}{}
\titlespacing{\chapter}{0pt}{0pt}{0pt}
\titlespacing{\section}{0pt}{20pt}{0pt}
\titlespacing{\subsection}{0pt}{20pt}{0pt}

% Настройка перечислений
\usepackage{enumitem}
\setlist[itemize]{label=\textendash, leftmargin=0mm, itemindent=18mm, itemsep=0mm, parsep=0mm, topsep=0cm}
\setlist[enumerate]{wide=0pt, leftmargin=0mm, itemindent=15mm, itemsep=0mm, parsep=0mm, topsep=0cm,label=\arabic*)}

% Настройка содержания
\usepackage{tocloft}
\renewcommand{\cftbeforechapskip}{0pt}
\renewcommand{\cftbeforetoctitleskip}{-10pt}
\renewcommand{\cftaftertoctitleskip}{8pt}
\renewcommand{\cfttoctitlefont}{\normalfont}
\renewcommand{\cftaftertoctitle}{\hfill}
\renewcommand{\cftchapdotsep}{2}
\renewcommand{\cftchappagefont}{\normalfont}
\renewcommand{\cftsecdotsep}{2}
\renewcommand{\cftsubsecdotsep}{2}
\renewcommand{\cftchapfont}{\normalsize}
\setlength{\cftsecindent}{1em}
\setlength{\cftsubsecindent}{2em}
\setlength{\cftsubsubsecindent}{3em}
\setlength{\cftchapnumwidth}{0.85em}
\setlength{\cftsecnumwidth}{1.7em}
\setlength{\cftsubsecnumwidth}{2.9em}

% Настройка листингов
\usepackage{listings}
\lstdefinestyle{mystyle}{	
	breakatwhitespace=false,         
	breaklines=true,                 
	captionpos=t,                    
	keepspaces=true,                 
	numbers=left,                    
	numbersep=15pt,                  
	showspaces=false,                
	showstringspaces=false,
	showtabs=false,                  
	tabsize=2
}
\lstset{style=mystyle}
\renewcommand{\ttdefault}{cmtt}
\lstset{basicstyle=\small\ttfamily}